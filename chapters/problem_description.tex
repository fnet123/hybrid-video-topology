\begin{titlingpage}

\noindent
\begin{tabular}{@{}p{4cm}l}
\textbf{Title:} 	& \thetitle \\
\textbf{Student:}	& \theauthor \\
\end{tabular}

\vspace{4ex}
\noindent\textbf{Problem description:}
\vspace{2ex}

Current video conferencing services use different topologies and architectures to realize real-time communication. One possible architecture, used by the service appear.in, is a full mesh architecture where each participant in a conversation has a full duplex connection with every other participant in the conversation. Another possible architecture, used by many traditional video conferencing services, is to use a Multipoint Control Unit (MCU). This unit will take in video and voice feeds sent from multiple participants which will then be combined into one video/voice feed that can be sent to all participants. This requires decoding and re-encoding of the streams in the MCU. A further possible architecture, is to send all streams through a central Selective Forwarding Unit (SFU), which will forward streams to select participants, based on available bandwidth and other preferences.

The different architectures for video conferencing have different properties. For example, for a conference with only two participants it is usually desirable to use the full mesh architecture, because the direct communication implies lower latency and better quality. However, with growing number of participants requirements on both bandwidth and CPU (due to extensive encoding and decoding) will imply that using full mesh becomes undesirable.  It is therefore likely that the optimal architecture uses a combination of different topologies, depending on the number of participants in a conversation and the resources available to each participant.

The task is to investigate what attributes and requirements a WebRTC conversation needs to perform optimally, and study how different topologies support these requirements.

\vspace{6ex}

\noindent
\begin{tabular}{@{}p{4cm}l}
\textbf{Responsible professor:} 	& \theprofessor \\
\textbf{Supervisor:}			& \thesupervisor \\
\end{tabular}

\end{titlingpage}
