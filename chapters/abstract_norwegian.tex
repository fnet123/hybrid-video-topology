\pagestyle{empty}
\begin{abstract}

Lite båndbreddebruk, lav forsinkelse, billig -- velg to. Dette er et kompromiss videokonferansetilbydere har måttet inngå så lenge de har vært bundet til en gitt nettverkstopologi. Topologien har satt grensene for hva som er mulig, og drift av et system med tilstrekkelig ytelse til at det adopteres av forbrukere har vært så dyrt at det kun er de største aktørene som kan konkurrere. Innovasjon er vanskelig i et system med så høye inngangsbarrierer, og både forbrukere og tjenestetilbydere lider som en konsekvens.

Denne oppgaven presenterer en hybrid-topologi for videokonferanser, som kan øke opplevd kvalitet med små ekstra kostnader. For eksisterende tilbydere som baserer seg på sentraliserte nettverk kan denne fremgangsmåten senke kostnader og forbedre ytelsen i mange tilfeller. For eksisterende tilbydere som er basert på jevnbyrdsnett kan metoden redusere tjenestens ressurskrav og øke ytelsen i mange situasjoner som er vanskelige i dag.

Systemet velger dynamisk den beste topologien for hver enkelt samtale, basert på egenskaper ved enhetene i samtalen. Videostrømmene vil rutes i nettet tilpasset hver enkelt enhets kapabiliteter. Løsningen kan utvides til å ta vilkårlige egenskaper ved enhetene og nettverket inn i beregningen, og kan benytte både supernoder, Selective Forwarding Units (SFU) og Multipoint Control Units (MCU) for å øke kvaliteten. Hver samtale modelleres som et multi-commodity flytnettverk og kan løses ved lineærprogrammering. Ikke-lineære egenskaper som køforsinkelse tilnærmes ved stykkevis lineære approksimasjoner.

Jevnbyrdsnettløsningen appear.in testes for å se hvordan videokonferanseløsninger over jevnbyrdsnett bygd på WebRTC yter, og for å illustrere potensialet for løsninger som kan overkomme begrensningene til en gitt topologi. Testene viser store ytelsesproblemer i Firefox på sterkt begrensede klienter, og et mer moderat potensiale for forbedring i Chrome. Verktøy for å bistå testingen ble utviklet og ligger vedlagt.

\end{abstract}
