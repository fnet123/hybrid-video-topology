\chapter{Background}
\label{chp:background}

In this chapter we'll first discuss some technical aspect of video conferencing that affect how we reason about solving the challenges, before performing we try to benchmark existing solutions to see how they perform in our example conversations.


\section{A technical look at video conferencing}

\subsection{Codecs}

\todo[inline]{Write about the future of H.264/VP9, SVC, hardware acceleration, etc}

\subsection{CPU limitations}

The naïve approach to encoding the videostreams is to encode the raw stream from the web camera into several client-optimized streams for transmissio
n. Using VP8, this is the only way to do it. However, H.264 can be encoded in a \gls{svc} manner, which makes it possible to extract different bandwid
th-streams from a single stream. With VP8 this is sadly not possible, and the only alternative is to utilize \gls{simulcast}, a technique that sends s
everal streams to the same endpoint, which can then select which of the streams to forward to the different endpoints. This is not as efficient as sen
ding only a single stream however, and the encoding step is also costlier on the CPU.

\subsection{Continuous Presence}

\todo[inline]{Write about Continuous Presence (multiple parties on-screen at the same time), and how that limits the different topologies}


\section{The current providers}

Now that we have established our example conversations, we can turn to see how they'll behave in the solutions available today. The most popular video conferencing solutions available today are:

\begin{center}
	\label{tab:existing-solutions}
	\begin{tabular}{| l | l |}
		\hline
		\textbf{Service} & \textbf{Description} \\ \hline
		appear.in & Peer-to-peer WebRTC service \\ \hline
		Google Hangouts & Browser-based service utilizing a Google MCU \\ \hline
		Microsoft Skype & Downloadable app, closed source protocol \\ \hline
		Cisco TelePresence & Custom hardware, self-hosted or cloud service \\ \hline
	\end{tabular}
\end{center}

Notably absent here is FaceTime, Apple's own videochat service. FaceTime's absence in this thesis is due to the lack of support for more than two people in a conversation, which makes the optimal topology question embarassingly easy to answer, and the lack of support on non-Apple devices, which makes the service uninteresting.

We also note that Mozilla just entered the market in collaboration with Telefonica with their Hello service, bundled with recent versions of Firefox\footnote{https://www.mozilla.org/en-US/firefox/hello/}. Hello however essentially provides the same service as appear.in, just bundled with the browser. Thus anything we say about appear.in applies to Firefox Hello as well, and we'll therefore not consider them separately. This author will however congratulate Mozilla and Telefonica on their endeavor to make communication more open and available to anyone.

Let's take a more detailed look at these services.

\subsection{appear.in}

\todo{Write appear.in background}


\subsection{Google Hangouts}

Google Hangouts is alongside appear.in the only other service based in the browser. Hangouts is a merge of several earlier Google communication solutions like Google Talk, Google+ Messenger and the Hangouts feature from Google+. The service uses a Google-provided \gls{mcu}, and VP8/9 over WebRTC as video transport\todo{Find a source to verify this}. A conversation is limited to 10 people.

To use Google Hangouts it requires you to setup a public Google+ profile, which makes the barrier of entry significantly higher than for some of the other services tested in this thesis. It also requires installing the Hangouts extension for video conferencing to work.

In our test scenarios, Google Hangouts performs like this:

\todo[inline]{Get test data from hangouts}


\subsection{Skype}

Skype is probably the most well-known of the solutions we're looking at, being the first to offer free video conferencing for personal use. Skype was also among the first to provide a VoIP solution interoperating with \gls{pstn}, easing the barrier of entry for new users. The original Skype topology was peer-to-peer, routing streams through PCs with Skype installed that had the app running, using a proprietary closed-source protocol. After the Microsoft aquisition Skype ditched the peer-to-peer design to a Microsoft MCU-backed solution, justified as a means to improve performance and security for users. A Skype conversation has a soft limit on five people for the best user experience, hard limited to 10 users.

Skype requires a standalone application to run, which is available on pretty much every platform out there, including Windows, Mac, Linux, Android, iOS, Windows Phone, BlackBerry, most tablets, TVs, gaming consoles and more. The benefit is close access to hardware and GPUs for more efficient video encoding, the downside is the lack of open protocols and standardization.

Skype performance summarized:

\todo[inline]{Add skype test data}


\todo{Which providers do we have today and what sort of topologies do they use? What are their pros and cons?}



\section{Where we want to be}

\todo{What is the intended outcome of this work?}


\section{Related Work}

\todo{What other work in this area have we learned from and built upon?}

A study at Chalmers in 2014\cite{tree-topology-webrtc} investigated the feasibility of utilizing normal nodes in a video conference as supernodes, routing traffic from less powerful nodes through these nodes to reduce network load. The authors conclude that such a solution is feasible given proper supernode selection, which gives even greater possibilities for a solution utilizing dynamic topologies like presented in this thesis. Pushing as much traffic as possible over client-provided supernodes lowers the cost for the provider, and enables better quality for the users since peers can be closer to each other than to the closest data center.
