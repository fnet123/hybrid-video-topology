\chapter{Discussion}
\label{chp:discussion}

In this chapter we'll discuss the feasibility of implementing the suggested solution, what considerations should be taken, and possible implications of this.


\section{Implementation}

\todo[inline]{Write about what's necessary to deploy this solution}

It must be noted, our custom algorithm will not be unjustly favored since it can access the additional forwarding nodes. Google Hangouts and Skype both utilize similar nodes, but their architecture and inner working is hidden from us due to the proprietary nature of these products, and thus they can not be modelled accurately. Only pure peer-to-peer WebRTC products like appear.in and Firefox Hello rely only on the client nodes, but they're both free to add any helpers themselves.



\section{Privacy}

For privacy-oriented consumers, the suggested solution opens up an interesting opportunity. Most of the existing solutions, like Google Hangouts and Skype, route every conversation through company-controlled datacenters, which in light of the NSA revelations is less than stellar from a privacy perspective. Dynamic topologies like discussed in this thesis, combined with a multitude of global VPS providers which provide quick-bootable disk images, enable consumers to provide their own infrastructure that can be used to offload their compute and connectivity to. Combined with WebRTC's mandatory encryption using DTLS, this makes conversations private and not susceptible to surveillance, from companies or governments.

For providers, this could be great news. They would not have to provide expensive infrastructure for call routing, and would rather focus on interfaces for finding friends and other complementary features like text chat, file sharing, call history, contact lists, etc.

This would make the market more volatile, as there would be a lower investment barrier to be able provide a full-blown communication infrastructure. New providers with original ideas could quickly blossom, as users traditionally have not shown a lot of loyalty to most communications platforms\footnote{We've seen this several times recently, as users have migrated en masse from SMS to Facebook Messenger to Whatsapp to Snapchat. The only value a user sees in their provider seem to be whether they can reach their friends through them.} Providing quick-bootable user-controlled server images could be a potential for innovation for companies. It boils down to where the user decides to put their trust, a powerful desktop at home could be used as a relay, or a cloud VPS could provide the same service, granted that you trust the company that provides them -- which does not have to be the same company that provides the room.

This author believe this decentralizes and opens up communication, which is more in line with the Internet spirit than the current walled-garden solutions.
